\documentclass[a4paper,12pt,english]{article}
\usepackage[utf8]{inputenc}
\usepackage{babel}
\usepackage{graphicx}
\usepackage{amsmath}
\usepackage{amssymb}
\usepackage{url}
\usepackage{hyperref}
\usepackage[left=3cm,top=2cm,right=3cm]{geometry}

\title{Prototypes are forever}
\author{Hugo Corbucci}
\date{}

\begin{document}

\maketitle

\section{Introduction}
With the growth of investments in software development, it is not
uncommon to encounter companies not related to software development
that want to build better tools for their business. Usually those
companies are not confident that the investment is good and wish to
invest as little as possible to understand whether the project might
be successful or not.

Due to the agile principle of delivering working software frequently
and the quest to quickly understand if a project is going to fail,
agile processes are very good candidates to develop those spike
projects. If the spike is successful, clients expect to evolve the
spike project to a full featured project without starting it all over.

On a prototyping development, adding spike features (ignoring several
known issues) and improving code quality to ensure a decent user
experience are antagonistic forces.

Investing in the code quality only repays after some time when
maintenance is a considerable part of the work. However, such
investment is useless if the project is aborted when the stakeholders
discover the project will not be lucrative. In such situation,
stakeholders pressure for new spike features. On the other hand,
developers feel that investment in quality is absolutely essential but
have trouble justifying it because maintenance might never be
necessary.

Once the project moves from spike/prototype project to a full feature
project, developers and stakeholders join to improve quality and
ensure all corner cases are covered. However such transition is
usually not very well delimited and the mindset change for the
stakeholders is quite fuzzy. Not to mention that if the first balance
is not fine tuned, the project might never reach the transition
stage. This report presents one case where the project came close to
the limit of complexity but managed to obtain the quality investment
and moved to a full featured project after eighteen months.

\section{Proposed report structure}

Introduction: plus de détails sur le projet
Prototypes et possibilités de problèmes.

Differentes phases du projet

Nos techniques pour le succes



``Prototypes are forever''?

Situation iniciale: Le prototype (4 personnes à mi-temps) - Domaine métier (Cenarios, Cenariste, etc)
Mise en établissement:
Changement de situation
Remise en établissement
Vis
Situation actuelle

Habituellement douloureux.
Prototypes restent pour toujours... Ce n'est pas forcément mauvais.
Solutions, techniques, tactiques qui permettent de survivre au temporaires.
Envolver equipe.

\begin{enumerate}
\item Starting a prototype project
\item Producing faulty features consciously
\item Changing the prototype's objective
\item Increasing complexity
\item Moving to a full featured project
\item Decreasing complexity and improving quality
\end{enumerate}

\section{About the author}

My name is Hugo Corbucci and I am a masters student at the University
of São Paulo (USP) in Brazil currently working to delineate the
proximity between open source software development and agile software
development. I am also one of the founders of Agilbits, a three years
old software development company in São Paulo, Brazil.

I can be contacted through my email at hugo@agilbits.com.br, mobile
phone at +55 11 8636 9029 or through post at:

Rua Trajano Reis, 185 - Ap 94 Bl 4

Jardim das Vertentes - São Paulo - SP

CEP: 05541-030

Brazil

\vspace{12pt} I was first in touch with agile software development in
2005 and have kept my interest ever since. From 2006 to 2009 I was
involved in the extreme programming laboratory course at USP first as
a student then as a teaching assistant. In 2007, I founded Agilbits
with four colleagues and a year later we got involved with the
mentioned project with a four people team keeping three to four weeks
releases but with hourly builds.

In July 2007, I co-founded the Coding Dojo São Paulo and have been
maintaining it ever since. Such work resulted in an experience report
published at Agile 2008 which I attended as well as Agile 2009. I am
also involved in the main agile event in São Paulo and I am working
with the Brazilian community to organise a national event in 2010.
\end{document}
