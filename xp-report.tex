% This is lnbip.tex the demonstration file of the LaTeX macro package for
% Lecture Notes in Business Information Processing from Springer-Verlag.
% It serves as a template for authors as well.
% version 1.0 for LaTeX2e
%
\documentclass[lnbip]{svmultln}
%
\usepackage{makeidx}  % allows for indexgeneration
% \makeindex          % be prepared for an author index
%
\newcommand{\mari}[1]{\footnote{MARI: #1}}
\begin{document}
%
\mainmatter              % start of the contribution
%
\title{Prototypes are forever\\
  Evolving from a prototype project to a full featured system}
%
\titlerunning{Prototypes are forever}  % abbreviated title (for running head)
%                                     also used for the TOC unless
%                                     \toctitle is used
%
\author{Hugo Corbucci\inst{1} and Mariana V. Bravo \inst{1}}
%
\authorrunning{Hugo Corbucci et al.}   % abbreviated author list (for running head)
%
%%%% list of authors for the TOC (use if author list has to be modified)
\tocauthor{Hugo Corbucci and Mariana V. Bravo}
%
\institute{Agilbits, Sao Paulo, Brazil,\\
\email{{hugo,marivb}@agilbits.com.br}}

\maketitle              % typeset the title of the contribution
% \index{Ekeland, Ivar} % entries for the author index
% \index{Temam, Roger}  % of the whole volume
% \index{Dean, Jeffrey}

\begin{abstract}        % give a summary of your paper

%TODO Abstract
This paper shows you how to handle prototype projects.
%                         please supply keywords within your abstract

%TODO Keywords
\keywords {prototype, agile methods, }
\end{abstract}
%
\section{Introduction}

Prototyping is one\mari{an} activity that most developer\mari{developers} have
heard about. Fred Brooks mentioned it in The Mythical Man-Month
\cite{Brooks1975} and most agile methods adopt various ideas related to
prototyping. Nevertheless, it is also one of the most feared\mari{"feared"
parece uma palavra estranha, meio extrema... é isso mesmo?} practices by those
who had some experience with it.

Most of the times\mari{Usually, in an Agile context}, using software prototypes
is traumatic because\mari{Talvez o ponto todo fique mais claro se vc inverter a
ordem: começa só contando a historia, e no final conclui que isso é traumatico}
successful prototypes look very much like full\mari{complete} features and
customers usually\mari{tira esse usually?} want to integrate the prototypes to
the working system and move on. However,\mari{The problem is that} prototypes
are frequently created in a ``quick and dirty'' fashion and the result is not
adequate to be incorporated in a full featured\mari{acho que deveria ser
"full-featured"} system. It is, however,\mari{prefiro "However, it is"} quite
hard to explain this fact to the stakeholders who usually do not want to spend
any more money\mari{do not want to invest any more} in this ``already working''
feature.

\mari{parece que nao deveria quebrar paragrafo aqui, vc continua contando a
mesma historinha} They, therefore, switch priorities and work efforts to other
parts of the system and let\mari{leave} the rough prototype lost within the code
base. Months or years later, the prototype becomes an importat\mari{important -
é isso mesmo que quis dizer? pq logo ali embaixo vc diz que ninguem sabe se é
mesmo importante} part of the system but is filled with bugs, unhandled corner
cases and, frequently, crappy code. Nobody remembers what it was supposed to do
or whether it is really important. Maintainability gets deeply affected and
developers have that natural and unpleasant I-told-you-so feeling.

Developers that have been through the pain of maintaing those dirty prototypes
hardly accept to work with prototypes anymore\mari{isso é mesmo verdade? me
parece meio forçado. talvez algo mais como "eles não querem"}. If they are
forced to, they make it so that there will be absolutely no way to integrate the
prototype to the existing system by either using a different platform, language
or even creating prototypes in other medias. That inflexibility can reduce the
capacity of responding to changes quickly and therefore harm the clients'
interests.

This paper presents how a four colocated people team\mari{four-people collocated
team} managed to start a prototyping project and evolve it naturally to a full
featured\mari{full-featured} software. The organization of this work follows a
chronological order as the project evolved softly from a series of prototyped
features to a full featured software\mari{vc acabou de falar que evolui de
prototipos pra full-featured, aí repete na frase seguinte. podia terminar essa
segunda frase apos o "order" ou "evolved".}. Section \ref{sec:start} will
present the project as the team received\mari{as the team first received it -
received parece estranho, que tal "as it was first presented to the team" ou
algo assim?} it. Section \ref{sec:working} presents the working system\mari{o
que é working system? sistema de trabalho? acho que algo como "work process" é
melhor} established by the team to create the software based on prototypes.
After some time, the team felt that the customer was shifting to a full featured
idea as described in Section \ref{sec:changes}. The following section (Section
\ref{sec:adapting}) shows how the team adapted to those changes to
accomodate\mari{acommodate} both ideas. Finally Section \ref{sec:nowadays}
presents the current status of the project and Section \ref{sec:conclusion}
concludes with a summary of practices that were useful to pass through this
experience without that much pain\mari{without much pain - eu tiraria o that}.

\section{Starting the project}
\label{sec:start}

Back in March 2008, Agilbits\mari{talvez seja melhor dizer "our company"? senão o cara fica se perguntando q q é a agilbits...} was hired to do some consulting with\mari{for} one of the biggest\mari{largest} movie producing companies in Brazil. The client had a fantastic\mari{great} idea for a software to write movie scripts and needed some help to evolve this idea. The company's job at the time was to scout the market, discover competitors and provide an estimation of the work needed to turn the client's idea into working software.\mari{Sugiro adicionar qual era a intencao do Paulo neste ponto. Algo como: His intention at that time was to build a working demo software as quickly as possible and use it to attract possible investors for a full program.}

For such work, one consultant was assigned to understand what were the client's needs and desires and two developers were asked to analyse\mari{analyze} the existing software\mari{the existing programs? desse jeito, nao sei se fica claro que sao mais de 1} and evaluate the possible development paths. After about 3 weeks of consulting and studies, the team handed a deck full of story cards estimated with the use of two platforms\mari{nao ficou claro. alternativa: story cards with two estimates each, based on the use of two possible platforms}. The first one\mari{platform} was an existing open source software with several features and a copyleft license. The second one was an Eclipse\mari{Eclipse ou eclipse? cada hora vc usa um...precisa escolher e revisar.} Rich Client Application\mari{developed from scratch} using Eclipse's open source framework\mari{fim de frase, tira o resto} as the basis for a new software.

That first estimation\mari{achei que o first deixou a frase confusa, pois vc acabou de enumerar as plataformas. sugiro trocar por initial. e talvez this ao inves de that} suggested that a four people team with a half-time dedication would be able to build a partially working prototype\mari{partially working prototype me parece uma expressao estranha. pq nao só "working prototype"?} of each feature described in about nine months of development using the existing open source software and about one year using eclipse's platform. The open source solution had the advantage to provide full functionality of several other features. For a full featured system\mari{full-featured... ou complete?}, the estimation was over\mari{was well over hehehe} 2 years of work on the Eclipse version and about a year and a half for the open source one.

After some discussion, the client opted for the eclipse based solution because his business plan was to keep the solution closed source and sell the project to external investors\mari{eu trocaria todo esse because por um "due to the license restriction of the open source one"}. He also chose to have just a prototype version\mari{to develop only a prototype of the idea ?} since 2 years seemed like a too heavy investment for him alone\mari{meio que ele ja tinha decidido isso antes? ou nao?}.

Such choice terminated\mari{frase fica estranha, terminate parece ruim, e alem disso qquer que fosse a escolha a consultoria terminaria. era a transicao natural. eu colocaria algo como "After this investigation phase..."} the consulting contract and created a new one with a majority of development time and a few hours of consulting\mari{acho que os detalhes de contrato q vc coloca nesse paragrafo sao meio desnecessarios, nao sei...}. Originally, this new contract mentioned a 4 developers team working on an open scope providing 160 hours of work each month. It specifically stated that the developers would work on pairs all the time and that the developed system should have automated tests to the production code.

The project goal was to create a software prototype with most faked (or simplified)\mari{pq o parentesis? deixa só "fake or simplified"} features and a few working ones. The client would use this prototype to present his ideas to some investors in Hollywood\mari{present his idea to investors - tira o "some" e o "in Hollywood"...} by October 2008. This meeting would either boost the project's development to a full featured system if the investors liked the idea or end its development in case they rejected the proposition\mari{rejected it}.

That was the team's vision of the project when the development begun. A short seven months project whose fate would be decided by its capacity to impress Hollywood investors\mari{impress investors}. Therefore, the main goal was to provide an excelent\mari{excellent} support for the client's show case\mari{demonstration} to ensure the project's growth and success. The next section (Section \ref{sec:working}) describes how the team organized itself to achieve this goal.

\section{Developing a work system}
\label{sec:working}

Given the project's situation, the customer was always pushing for new
features as fast as possible considering only one specific use
scenario. This meant that, for most features, there were several cases
which the team was asked \textbf{not} to handle. Regarding the source
code, this meant a lot of conditionals, several spikes becoming
permanent solutions and a fair amount of unhandled exceptions, ignored
errors or non functional regular behaviors.

The team knew since the beginning that the client would change his
mind over time. After all, it was partly to better understand his idea
and its aplicability that he wanted to build this prototype. So things
were going to change and features would be developed to later be
thrown away while code produced only for a quick spike was going to
become part of the system. Therefore, the team started investing a
little on design, automated tests and refactoring since the beginning
and made it clear for the client that there would be some work done on
features after he accepted them to polish the work.

The first few month went quite smooth. The main features developed
were to integrate with a text format coming from competitor software,
provide a simple text marking feature and a visualization feature to
manipulate and visualise the marks. For those features, it was quite
simple to avoid gaps since there were not many business rules
involved. Problems started to appear once the script writting business
rules started to show up.

The client's presentation script was evolving as the software did and
the team soon started to add conditionals to ignore cases he would not
enter in. By October, the main features were ready but new discovered
features were still incipient and the client was not feeling confident
to present the software to the investors. However, he started to make
contact with a few people to schedule a meeting by the end of November
2008 and December 2008. Those dates became our new deadline until
which all efforts should be focused in making those incipient features
available for the demonstration.

A new feature pressure installed itself since the projects fate would
be decided at the demonstration and it was close! The customer wanted
the team to ignore corner cases, speed up delivery and ensure the
demonstration would run smoothly. The excitment from the important
presentation to other people let the development team highly motivated
to deliver all features the client had asked for. Although unit
testing and pair programming was a mandatory rule on the team, the
general will to quickly deliver the features decreased the code
quality.

\section{Changing the rules}
\label{sec:changes}

December 2008 arrived and passed without any meeting. The company that
the client was in contact with had just been acquired by another one
so any project presentation was useless until things settled
down. That news pushed the deadline away for another 3 or 4 months at
least. Along with this news came the information that the client had
formed a dramaturgy experts group to help him understand better how to
structure the software.

This new context relieved a 4 months pressure of upcoming deadline
over a team which was beginning to feel the burden of unhandled
technical debt. All members of the development team agreed that the
code was getting complex and the quality was decreasing which was
affecting productivity and speed. The software was now going to have a
set of beta testers and it needed to perform decently to allow the
users to suggest improvements in the work system.

The general feeling was that the project was no longer aimed at a
simple presentation to investors. It was softly switching to a more
elaborate and end user oriented software. The current development
approach would not be able to support this new use of the system. The
change had to be clear to the client so that development efforts would
be directed to address this new way of working.

The warnings came surely from the dramaturgy study group. They started
having troubles with several known and unknown corner cases,
behaviours and just plain old bugs. The client started to notice that
the users were having several troubles with the software and decided
we needed to invest more in usability and user experience. To which
the team replied that it would also mean less new features.

At this point, the client started to understand the dilema that the
developpers had felt so far. How to keep a good rhythm of new features
and still cover most use cases of existing features? Another critical
issue in the software was that, so far, most features aimed at
visualization and insertion of meta data in the movie script but users
were claiming for basic text editing features ignored so far.

The team estimated that to have an editor with the basic features
expected by the client would take at least three full iterations. This
was completly unacceptable to the client since it would mean no new
feature until the moment when he would possibly be able to show the
software to investors.

\section{Adapting to the new rules}
\label{sec:adapting}

\section{Current status}
\label{sec:nowadays}

\section{Conclusion}
\label{sec:conclusion}

%
% ---- Bibliography ----
%
\begin{thebibliography}{5}

\bibitem{smit:wat} Smith, T.F., Waterman, M.S.: Identification of Common Molecular
Subsequences. J. Mol. Biol. 147, 195--197 (1981)

\bibitem{mes} May, P., Ehrlich, H.C., Steinke, T.: ZIB Structure Prediction Pipeline:
Composing a Complex Biological Workflow through Web Services. In: Nagel,
W.E., Walter, W.V., Lehner, W. (eds.) Euro-Par 2006. LNCS, vol. 4128,
pp. 1148--1158. Springer, Heidelberg (2006)

\bibitem{fos:kes} Foster, I., Kesselman, C.: The Grid: Blueprint for a New Computing
Infrastructure. Morgan Kaufmann, San Francisco (1999)

\bibitem{cff} Czajkowski, K., Fitzgerald, S., Foster, I., Kesselman, C.: Grid
Information Services for Distributed Resource Sharing. In: 10th IEEE
International Symposium on High Performance Distributed Computing, pp.
181--184. IEEE Press, New York (2001)

\bibitem{fos:kes:2} Foster, I., Kesselman, C., Nick, J., Tuecke, S.: The Physiology of the
Grid: an Open Grid Services Architecture for Distributed Systems
Integration. Technical report, Global Grid Forum (2002)

\bibitem{url} National Center for Biotechnology Information, http://www.ncbi.nlm.nih.gov

\end{thebibliography}
%
\end{document}
