% This is lnbip.tex the demonstration file of the LaTeX macro package for
% Lecture Notes in Business Information Processing from Springer-Verlag.
% It serves as a template for authors as well.
% version 1.0 for LaTeX2e
%
\documentclass[lnbip]{svmultln}
%
\usepackage{makeidx}  % allows for indexgeneration
% \makeindex          % be prepared for an author index
%
\begin{document}
%
\mainmatter              % start of the contribution
%
\title{Prototypes are forever\\
  Evolving from a prototype project to a full featured system}
%
\titlerunning{Prototypes are forever}  % abbreviated title (for running head)
%                                     also used for the TOC unless
%                                     \toctitle is used
%
\author{Hugo Corbucci\inst{1} and Mariana V. Bravo \inst{1}}
%
\authorrunning{Hugo Corbucci et al.}   % abbreviated author list (for running head)
%
%%%% list of authors for the TOC (use if author list has to be modified)
\tocauthor{Hugo Corbucci}
%
\institute{Agilbits, Sao Paulo, Brazil,\\
\email{hugo@agilbits.com.br}}

\maketitle              % typeset the title of the contribution
% \index{Ekeland, Ivar} % entries for the author index
% \index{Temam, Roger}  % of the whole volume
% \index{Dean, Jeffrey}

\begin{abstract}        % give a summary of your paper
This paper shows you how to handle prototype projects.
%                         please supply keywords within your abstract
\keywords {prototype, agile methods, }
\end{abstract}
%
\section{Introduction}

%TODO Review the uses of first person and third person.
Prototyping is one activity that most developer have heard
about. However it is also one of the most feared ones by those who had
some experience with it. The experience is, most of the time,
traumatic because developers end up having to integrate the prototype
to the final product and maintain it. Developers that have been
through the pain to maintain and distribute what once was a prototype
hardly accept to work with prototypes anymore. If they are forced to,
they make it so that there will be absolutely no way to integrate the
prototype to the existing system by either using a different platform,
language or even creating prototypes in other medias.

The main reported issue with software prototypes is that, if they are
successful, the clients/stakeholders/users do not see why more effort
should be invest in the prototypes. They, therefore, switch priorities
and work efforts to other parts of the system and let the rough
prototype lost within the system. Months or years later, the prototype
becomes an importat part of the system but is filled with bugs,
unhandled corner cases and, most of the time, crappy code. Nobody
remembers what it was supposed to do or whether it is really
important. Maintainability gets deeply affected and developers have
that natural and unpleasant I-told-you-so feeling.

This paper presents how a four colocated people team managed to start
a prototyping project and evolve it naturally to a full featured
software. The organization of this work follows a chronological order
as the project evolved softly from a series of prototyped features to
a full featured software. Section \ref{sec:start} will present the
project as the team received it. Section \ref{sec:working} presents
the working system established by the team to create the software
based on prototypes. After some time, the team felt that the customer
was shifting to a full featured idea as described in Section
\ref{sec:changes}. The following section (Section \ref{sec:adapting})
shows how the team adapted to those changes to accomodate both
ideas. Finally Section \ref{sec:nowadays} presents the current status
of the project and Section \ref{sec:conclusion} concludes with a
summary of practices that were useful to pass through this experience
without that much pain.

\section{Starting the project}
\label{sec:start}

Back in March 2008, Agilbits was hired to do some consulting with one
of the biggest movie producing companies in Brazil. The client had a
fantastic idea for a software to write movie scripts and needed some
help to evolve this idea. The company's job at the time was to scout
the market, discover competitors and provide an estimation of the work
needed to turn the client's idea into working software.

For such work, the company assigned one consultant to understand what
were the client's needs and desires and two developers to analyse the
existing software and evaluate the possible development paths. After
about 3 weeks of consulting and studies, we handed a deck full of
story cards estimated with the use of two platforms. The first one, an
existing open source software with several features and a copyleft
license. The second one, an Eclipse Rich Client Application using
Eclipse's open source framework as the basis for a new software.

Our first estimation suggested that a four people team with a
half-time dedication would be able to build a partially working
prototype of each feature described in about nine months of
development using the existing open source software and about one year
using eclipse's platform. For a full featured system, the estimation
was over 2 years of work. Our client opted for the eclipse based
solution because his investment plan was to keep the solution closed
source and sell the project to external investors. He also chose to
have just a prototype version since 2 years seemed like a too heavy
investment for him alone. With such choice, the contract changed from
a consulting one to a development one.

The client hired the company to provide a 4 developers team working on
an open scope providing 160 hours of work each month. Our contract
specificaly stated that the developers would work on pairs all the
time and we were supposed to handle automated tests along with the
production code. The project goal was to create a software prototype
with most faked (or simplified) features and a few working ones. Our
client would use this prototype to present his ideas to some investors
by October 2008 and therefore either end the project or boost its
development to a full featured system depending of the outcome of the
meeting with investors.

That was our vision of the project when the development begun. Our
target was to provide an excelent support for the show case our client
would have to gather investments. The next section (Section
\ref{sec:working}) describes how we organized to handle such plan.

\section{Developing a work system}
\label{sec:working}

Given the environment we were on, the customer was always pushing for
new features as fast as possible considering only one specific use
scenario. This meant that, for most features, there were several cases
which we were asked not to handle. Regarding the source code, this
meant a lot of conditionals, several spikes becoming permanent
solutions and a fair amount of unhandled exceptions, errors or just
regular behaviors.

We knew since the beginning that our client would change his mind over
time. After all, it was partly to better understand his idea and its
aplicability that he wanted to build this prototype. So things were
going to change and we would throw features away and keep code meant
only for a quick spike. Therefore, we started investing a little on
design, automated tests and refactoring since the beginning and we
made it clear for our client that there would be some work done on
features after he accepted them.

The first few month went quite smooth. The main features involved were
to integrate wth a text format coming from competitors, provide a
simple text marking feature and a visualization feature to manipulate
and visualise the marks. For those features, it was quite simple to
avoid gaps since there were not many  business rules
involved. Problems started to appear once the script writting business
rules started to show up.

Our client's presentation script was evolving as the software did and
we soon started to add conditionals to ignore cases he would not enter
in. By October, the main features were ready but new features were
still incipient and our client was not feeling confident to present
the software to the investors. However, he started to make contact
with a few people to schedule a meeting by the end of November 2008
and December 2008. Those dates became our new deadline until which all
efforts should be focused in making those incipient features available
for the demonstration.

A new feature pressure installed itself since the projects fate would
be decided at the demonstration. The customer wanted the team to
ignore corner cases, speed up delivery and ensure the demonstration
would run smoothly. The excitment from the important presentation to
other people let the development team highly motivated to deliver all
features the client had asked for. Although unit testing and pair
programming was a mandatory rule on the team, the general will to
quickly deliver decreased the code quality.

\section{Changing the rules}
\label{sec:changes}

December 2008 arrived and no meeting was scheduled. Our client told us
that the company he was going to contact had just been bought so any
project presentation was useless until things settled down. Our so
expected deadline was then pushed away for another 3 or 4 months at
least. Along with this news came the information that our client had
formed a dramaturgy experts group to help him understand the best way
to build the software.

The pressure release from the last 2 months, this new data plus a
known increase in code complexity and decrease in quality led the
team to re-evaluate their working system.

\section{Adapting to the new rules}
\label{sec:adapting}

\section{Current status}
\label{sec:nowadays}

\section{Conclusion}
\label{sec:conclusion}

%
% ---- Bibliography ----
%
\begin{thebibliography}{5}

\bibitem{smit:wat} Smith, T.F., Waterman, M.S.: Identification of Common Molecular
Subsequences. J. Mol. Biol. 147, 195--197 (1981)

\bibitem{mes} May, P., Ehrlich, H.C., Steinke, T.: ZIB Structure Prediction Pipeline:
Composing a Complex Biological Workflow through Web Services. In: Nagel,
W.E., Walter, W.V., Lehner, W. (eds.) Euro-Par 2006. LNCS, vol. 4128,
pp. 1148--1158. Springer, Heidelberg (2006)

\bibitem{fos:kes} Foster, I., Kesselman, C.: The Grid: Blueprint for a New Computing
Infrastructure. Morgan Kaufmann, San Francisco (1999)

\bibitem{cff} Czajkowski, K., Fitzgerald, S., Foster, I., Kesselman, C.: Grid
Information Services for Distributed Resource Sharing. In: 10th IEEE
International Symposium on High Performance Distributed Computing, pp.
181--184. IEEE Press, New York (2001)

\bibitem{fos:kes:2} Foster, I., Kesselman, C., Nick, J., Tuecke, S.: The Physiology of the
Grid: an Open Grid Services Architecture for Distributed Systems
Integration. Technical report, Global Grid Forum (2002)

\bibitem{url} National Center for Biotechnology Information, http://www.ncbi.nlm.nih.gov

\end{thebibliography}
%
\end{document}
