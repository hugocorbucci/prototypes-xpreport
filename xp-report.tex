% This is lnbip.tex the demonstration file of the LaTeX macro package for
% Lecture Notes in Business Information Processing from Springer-Verlag.
% It serves as a template for authors as well.
% version 1.0 for LaTeX2e
%
\documentclass[lnbip]{svmultln}
%
\usepackage{makeidx}  % allows for indexgeneration
% \makeindex          % be prepared for an author index
%
\newcommand{\mari}[1]{\footnote{MARI: #1}}
\begin{document}
%
\mainmatter              % start of the contribution
%
\title{Prototypes are forever\\
  Evolving from a prototype project to a full featured system}
%
\titlerunning{Prototypes are forever}  % abbreviated title (for running head)
%                                     also used for the TOC unless
%                                     \toctitle is used
%
\author{Hugo Corbucci\inst{1} and Mariana V. Bravo \inst{1}}
%
\authorrunning{Hugo Corbucci et al.}   % abbreviated author list (for running head)
%
%%%% list of authors for the TOC (use if author list has to be modified)
\tocauthor{Hugo Corbucci and Mariana V. Bravo}
%
\institute{Agilbits, Sao Paulo, Brazil,\\
\email{{hugo,marivb}@agilbits.com.br}}

\maketitle              % typeset the title of the contribution
% \index{Ekeland, Ivar} % entries for the author index
% \index{Temam, Roger}  % of the whole volume
% \index{Dean, Jeffrey}

\begin{abstract}        % give a summary of your paper

%TODO Abstract
This paper shows you how to handle prototype projects.
%                         please supply keywords within your abstract

%TODO Keywords
\keywords {prototype, agile methods, }
\end{abstract}
%
\section{Introduction}

Prototyping is an activity that most developers have heard about. Fred
Brooks mentioned it in The Mythical Man-Month \cite{Brooks1975} as one
of the best ways to provide a quick view of a feature to the clients
or users to help them make a choice. Dynamic System Development Method
(DSDM) is heavily based on prototyping and other agile methods also
adopt many ideas related to it. Nevertheless, it is also one of the
most feared\mari{"feared" parece uma palavra estranha, meio
  extrema... é isso mesmo?} practices by those who had some experience
with it.

Successful software prototypes look very much like complete features
given a certain execution path. Therefore it is common that the
customers get so happy with it that they want to integrate the
prototypes to the working system and move on. The problem is that
prototypes are frequently created in a ``quick and dirty'' fashion and
the result is not adequate to be incorporated in a full-featured
system. However, it is quite hard to explain this fact to the
stakeholders who usually do not want to invest any more money in this
``already working'' feature. The consequence is that they switch
priorities and work efforts to other parts of the system and leave the
rough prototype lost within the code base. Months or years later, the
prototype becomes a part of the system but is filled with bugs,
unhandled corner cases and, frequently, crappy code. Nobody remembers
what it was supposed to do or whether it is really
important. Maintainability gets deeply affected and developers have
that natural and unpleasant I-told-you-so feeling which results in a
traumatic feeling.

Developers that have been through the pain of maintaing those dirty
prototypes are not entusiastic to work with prototypes anymore. If
they are forced to, they make it so that there will be absolutely no
way to integrate the prototype to the existing system by either using
a different platform, language or even creating prototypes in other
medias. That inflexibility can reduce the capacity of responding to
changes quickly and therefore harm the clients' interests.

This paper presents how a four-people collocated team managed to start
a prototyping project and evolve it naturally to a full-featured
software. The organization of this work follows a chronological order
as the project evolved. Section \ref{sec:start} will present the
project as it was first presented to the development team. Section
\ref{sec:working} presents the work process established by the team to
create the software based on prototypes.  After some time, the team
felt that the customer was shifting to a full featured idea as
described in Section \ref{sec:changes}. The following section (Section
\ref{sec:adapting}) shows how the team adapted to those changes to
acommodate both ideas. Finally Section \ref{sec:nowadays} presents the
current status of the project and Section \ref{sec:conclusion}
concludes with a summary of practices that were useful to pass through
this experience without much pain.

\section{Starting the project}
\label{sec:start}

Back in March 2008, our company was hired to do some consulting for
one of the largest movie producing companies in Brazil. The client had
a great idea for a software to write movie scripts and needed some
help to evolve this idea. The company's job at the time was to scout
the market, discover competitors and provide an estimation of the work
needed to turn the client's idea into working software.\mari{Sugiro
  adicionar qual era a intencao do Paulo neste ponto. Algo como: His
  intention at that time was to build a working demo software as
  quickly as possible and use it to attract possible investors for a
  full program.}

For such work, one consultant was assigned to understand what were the
client's needs and desires and two developers were asked to analyze
the existing script writing programs and evaluate the possible
development paths. After about 3 weeks of consulting and studies, the
team handed a deck full of story cards with two estimates each, based
on the use of two possible platforms. The first platform was an
existing open source software with several features and a copyleft
license. The second one was an Eclipse Rich Client Application
developed from scratch using Eclipse's open source framework.

This initial estimation suggested that a four people team with a
half-time dedication would be able to build a working prototype of
each feature described in about nine months of development using the
existing open source software and about one year using Eclipse's
platform. The open source solution had the advantage to provide full
functionality of several other features. For a complete system, the
estimation was well over 2 years of work on the Eclipse version and
about a year and a half for the open source one.

After some discussion, the client opted for the Eclipse based solution
due to the license restriction of the open source one which conflicted
with his business plan. He also chose to develop only a prototype of
the idea since 2 years seemed like a too heavy investment for him
alone\mari{meio que ele ja tinha decidido isso antes? ou nao?}.

After the investigation pahse the consulting contract and created a
new one with a majority of development time and a few hours of
consulting\mari{acho que os detalhes de contrato q vc coloca nesse
  paragrafo sao meio desnecessarios, nao sei...}. Originally, this new
contract mentioned a 4 developers team working on an open scope
providing 160 hours of work each month. It specifically stated that
the developers would work on pairs all the time and that the developed
system should have automated tests to the production code.

The project goal was to create a software prototype with most faked or
simplified features and a few working ones. The client would use this
prototype to present his ideas toinvestors by October 2008. This
meeting would either boost the project's development to a full
featured system if the investors liked the idea or end its development
in case they rejected it.

That was the team's vision of the project when the development
begun. A short seven months project whose fate would be decided by its
capacity to impress investors. Therefore, the main goal was to provide
an excellent support for the client's demonstration to ensure the
project's growth and success. The next section (Section
\ref{sec:working}) describes how the team organized itself to achieve
this goal.

\section{Developing a work system}
\label{sec:working}

Given the project's situation, the customer was always pushing for new
features as fast as possible considering only one specific use
scenario. This meant that, for most features, there were several cases
which the team was asked \textbf{not} to handle. Regarding the source
code, this meant a lot of conditionals, several spikes becoming
permanent solutions and a fair amount of unhandled exceptions, ignored
errors or non functional regular behaviors.

The team knew since the beginning that the client would change his
mind over time. After all, it was partly to better understand his idea
and its aplicability that he wanted to build this prototype. So things
were going to change and features would be developed to later be
thrown away while code produced only for a quick spike was going to
become part of the system. Therefore, the team started investing a
little on design, automated tests and refactoring since the beginning
and made it clear for the client that there would be some work done on
features after he accepted them to polish the work.

The first few month went quite smooth. The main features developed
were to integrate with a text format coming from competitor software,
provide a simple text marking feature and a visualization feature to
manipulate and visualise the marks. For those features, it was quite
simple to avoid gaps since there were not many business rules
involved. Problems started to appear once the script writting business
rules started to show up.

The client's presentation script was evolving as the software did and
the team soon started to add conditionals to ignore cases he would not
enter in. By October, the main features were ready but new discovered
features were still incipient and the client was not feeling confident
to present the software to the investors. However, he started to make
contact with a few people to schedule a meeting by the end of November
2008 and December 2008. Those dates became our new deadline until
which all efforts should be focused in making those incipient features
available for the demonstration.

A new feature pressure installed itself since the projects fate would
be decided at the demonstration and it was close! The customer wanted
the team to ignore corner cases, speed up delivery and ensure the
demonstration would run smoothly. The excitment from the important
presentation to other people let the development team highly motivated
to deliver all features the client had asked for. Although unit
testing and pair programming was a mandatory rule on the team, the
general will to quickly deliver the features decreased the code
quality.

Things were getting quite unpleasant from a development perspective
but since the client satisfaction was still high, there was little
that could be done. However, external interference was about to change
a bit the situation. Section \ref{sec:changes} will explain how the
project got affected and what new direction those changes pointed to.

\section{Changing the rules}
\label{sec:changes}

December 2008 arrived and passed without any meeting. The company that
the client was in contact with had just been acquired by another one
so any project presentation was useless until things settled
down. That news pushed the deadline away for another 3 or 4 months at
least. Along with this news came the information that the client had
formed a dramaturgy experts group to help him understand better how to
structure the software.

This new context relieved a 4 months pressure of upcoming deadline
over a team which was beginning to feel the burden of unhandled
technical debt. All members of the development team agreed that the
code was getting complex and the quality was decreasing which was
affecting productivity and speed. The software was now going to have a
set of beta testers and it needed to perform decently to allow the
users to suggest improvements in the work system.

The general feeling was that the project was no longer aimed at a
simple presentation to investors. It was softly switching to a more
elaborate and end user oriented software. The current development
approach would not be able to support this new use of the system. The
change had to be clear to the client so that development efforts would
be directed to address this new way of working.

The warnings came quickly from the dramaturgy study group. They
started having troubles with several known and unknown corner cases,
behaviours and just plain old bugs. The client started to notice that
the users were having several troubles with the software and decided
we needed to invest more in usability and user experience. To which
the team replied that it would also mean less new features.

At this point, the client started to understand the dilema that the
developpers had felt so far. How to keep a good rhythm of new features
and still cover most use cases of existing features? Another critical
issue in the software was that, so far, most features aimed at
visualization and insertion of meta data in the movie script but users
were claiming for basic text editing features ignored so far.

The team estimated that to have an editor with the basic features
expected by the client would take at least three full iterations. This
was completly unacceptable to the client since it would mean no new
feature until the moment when he would possibly be able to show the
software to investors. So he took another action which indicated
changes in the business plan of the project. He decided he wanted to
have another team working on an editor feature that would allow all
the features he wanted.

The team did some research and ended up discovering an open source
Eclipse Rich Client WYSIWYG (What You See Is What You Get) HTML
editor\footnote{\url{http://onpositive.com/richtext/} - Last accessed on
20/02/2010}. The editor relied on a reimplementation of Eclipse's
StyledText component which is responsible for rendering text within
Eclipse editors. This kind of knowledge was close enough from the one
needed to implement the features our client wanted on our
application. So the client outsourced the development of this
underlying infra-structure as a way to keep the feature producing
speed while attending the users' requests.

Meanwhile, the development team was concerned with the increasing
complexity so they started to track some data from the source code to 

\section{Adapting to the new rules}
\label{sec:adapting}

Investing in reducing instanceof uses, TODOs and FIXMEs as well as
increasing line of tests. Noticing most issues are coming from
UI. SWTBot?

Effort dedication to integrate the new outsourced component developed.

Always handle (even if it means throwing an error to the log) corners
cases or unexpected input/paths. Always refactor before completing any
task. Tolerance 0 to bugs. Bug means highest priority.

Introducing another developer in the team.

\section{Current status}
\label{sec:nowadays}

Test coverage

Testing in other platforms

Merciless decoupling

\section{Conclusion}
\label{sec:conclusion}

Team's effort to keep tests level (ignoring known issues not solved).

Refactor tests.

Refactor prototypes (their code must be good).



%
% ---- Bibliography ----
%
\begin{thebibliography}{5}

\bibitem{Brooks1975} Brooks Jr., F.P.: The Mythical Man Month: Essays
  on Software Engineering. Addison-Wesley (1975)

\end{thebibliography}
%
\end{document}
